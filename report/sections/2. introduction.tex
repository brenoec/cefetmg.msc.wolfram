
\section{Introdução}

Esta seção apresenta breve introdução ao Automato de Wolfram e suas quatro
classes de evolução. A seção seguinte apresenta os materiais e método, seguido
dos resultados, uma breve discussão.

\subsection{Automato de Wolfram}

O Automato de Wolfram \cite{2002:Wolfram:New:Science} é um automato
unidimensional com $n$ sítios e $2$ estados possíveis ${0, 1}$. O estáto
fututo de um sítio depende de seus $2$ vizinhos imediatos:

\begin{equation}
  x_{t+1} = r_i(x_t-1, x_t, x_t+1).
\end{equation}

\vspace{5mm}
A tripla de sítios tem $2^3 = 8$ estados possíveis, já que cada sítio pode
assumir $2$ valores. Como cada tripla pode levar a $2$ estados possíveis, temos
um total de $2^8 = 256$ regras.

\vspace{5mm}
É prático e comum se referir a uma regra através da conversão da sequência de
bits para base decimal. Assim como é prático e comum determinar a regra a partir
de sua referência. A regra 00011110 é denominada regra 30, por exemplo, pois
$00011110_2 = 30_{10}$.

\begin{equation}
  r_{30} = 00011110.
  \label{eq:rule_30}
\end{equation}

\vspace{5mm}
Para entender o que a equação \ref{eq:rule_30} significa, lê-se as relações
abaixo da seguinte maneira: a tripla referente ao sítio $x$ à esquerda faz com
que o próximo estado de $x$ seja o valor à direita:

\begin{equation}
  \begin{aligned}
    111 \rightarrow 0 \\
    110 \rightarrow 0 \\
    101 \rightarrow 0 \\
    100 \rightarrow 1 \\
    011 \rightarrow 1 \\
    010 \rightarrow 1 \\
    001 \rightarrow 1 \\
    000 \rightarrow 0
  \end{aligned}
  \label{eq:rule_30_values}
\end{equation}

\vspace{5mm}
O número binário formado pelos valores das parcelas da direita, de cima para
baixo, na equação \ref{eq:rule_30_values}, convertido para base decimal é 30.

\subsection{Classes de Evolução}

São quatro as classes de evolução possíveis para o Automato de Wolfram
\cite{Atman:2002:Sistemas:Complexos}:

\begin{itemize}
  \item Evolução Homogênea: sítios permanecem em determinados estados de modo
  estacionário, de modo que o sistema evolui para o mesmo estado a cada
  iteração;

  \item Evolução de Estruturas Periódicas: sítios alternam estados
  periodicamente, de modo que a evolução do sistema é periódica;

  \item Evolução Caótica: o sistema evolui de forma desordenada e não apresenta
  padrões;

  \item Evolução Complexa: o sistema apresenta estruturas complexas que evoluem
  sem previsibilidade.
\end{itemize}
